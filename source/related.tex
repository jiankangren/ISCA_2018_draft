\section{Related Work} \label{sec:related}
%%\vspace{-5pt}
%
To support the task progress under intermittent power supply, both software and hardware recovery mechanisms have been proposed to support automatic recovery of the processor.

%\vspace{5pt}
\noindent\textbf{Software Mechanism: Checkpointing.} \\
%
Checkpointing is a software based recovery mechanism for processors.
Checkpoints are placed in the program, where the processor state are stored into non-volatile memories.
By rolling back to these checkpoints, the processor can keep the task progress after power failure~\cite{Dong2011Hybrid, Ma2015Architecture}.
Bronevetsky et al. propose the application level checkpointing strategy on the shared memory systems to enhance the reliability~\cite{Bronevetsky2004Application}.
Mementos proposes the concept of transiently powered computer and presents a software processing strategy which transforms the general purpose programs into interruptible computations for the general hardware architecture with Flash memory~\cite{ransford2012mementos}.
Balsamo et al. propose Hibernus and Hibernus++ using FeRAM and an interrupt-based checkpointing solution to reduce the performance overhead~\cite{balsamo2015hibernus,Balsamo2016Hibernus++,Rodriguez2015Approaches}.
Jayakumar et al. design QuickRecall and em-Map to integrate FeRAM into main memory to reduce the backup data size and lower the failure voltage threshold~\cite{jayakumar2014quickrecall,jayakumar2015q}.
In addition to general purpose processor architectures, Mirhoseini et al. target the Application Specific Integrated Circuits (\emph{ASICs}) and propose Idetic and Chime with the help of control data flow graphs (\emph{CDFGs})~\cite{Mirhoseini2013Idetic,Mirhoseini2013Automated,Mirhoseini2016Chime}.

These schemes achieve continuous progress of processors with out peripherals.
Large rollback overhead may take place while recovering a multi-device system with these strategies. 

%\vspace{5pt}
\noindent\textbf{Hardware Mechanism: Non-volatile Processor.} \\
%
Besides the checkpointing mechanisms, researchers also provide solutions in hardware domain.
Non-volatile processor draws a lot of attention due to its ability to store the system state and data automatically in hardware.
The first processor chip designed by Wang et al. using FeRAM realizes the ability to backup and restore the processor state and data within $3\mu s$~\cite{wang20123us}.
Bartling et al. propose a non-volatile logic based Cortex-M0 chip with higher performance and lower leakages~\cite{Bartling2013An}.
Sakimura et al. from NEC propose the non-volatile magnetic flip-flops~\cite{Sakimura2009Nonvolatile} and a 20MHz non-volatile micro-controller with STT-RAM~\cite{Sakimura201410}.
Recently, Liu et al. propose an enhanced NVP based on ReRAM which has the highest integration level~\cite{liu2016a}.
In addition, Li et al. propose the non-volatile I/O (NVIO) enabling efficient automatic reconfiguration of  I/O interfaces~\cite{li2016hw}.

Compared with software checkpointing strategies, non-volatile devices enable state recovery with higher speed and fewer rollbacks.
However, these nonvolatile processors lack of flexibility on checkpointing and recovery which may cause inconsistency problems in a multi-device system.


%\vspace{5pt}
\noindent\textbf{Inconsistency and Program Partitioning.} \\
%
Researchers have noticed the problem of the inconsistency issues while rolling back a system with nonvolatile memories.
B. Lucia et al.~\cite{Lucia2015} discover and model the data inconsistency problem caused by improper rollbacks after power failures.
After that, more works~\cite{van2016intermittent}, ~\cite{colin2016chain}, ~\cite{Xie2015} analyze the scenarios of checkpoint recoveries and propose careful checkpointing strategies to ensure the correctness of all checkpoints.
Recently, an energy-interference-free debugger for intermittent powered systems is proposed by A. Colin et al. to provide a more reliable and convenient debugging platform~\cite{Colin2016An}.
However, reliabliliy and efficiency challenges still exist when adopting NVP in a TPC system with multiple peripherals and interrupts.

%These works analyzes the transactional program in a TPC system and proposes proper program partitioning methods to avoid sematics issues caused by inconsistency.
%However, the consistency of the processor and the multiple peripherals raises new challenges.
%Therefore, REMARK is proposed target on the efficiency and reliability challenges in the recovery of multi-device TPC systems.
