\section{Introduction} \label{sec:introduction}
%   1.5pages
% Transiently Powered Systems and Sustaining computing.
Transiently Powered Computers (TPCs)~\cite{Ma2015Architecture, ransford2013transiently, ransford2012mementos} are a new class of battery-less embedded systems that depend solely on energy harvested from ambient environment.
TPC systems are especially attractive to many IoT applications since it is often challenging to employ traditional batteries where it is inconvenient, costly or even dangerous to replace or service batteries. 
Energy harvesting can eliminate the need for batteries or wires and enable long-term operation of these systems with little maintenance.
However, design of these systems faces unique challenges since the power supply is intrinsically unstable and affected by the ambient environment conditions~\cite{Ma2015Architecture, ransford2013transiently,wang2014mppt}.
For example, the reliability of solar powered systems largely depends on the weather, location and environmental changes of different applications~\cite{Abas2015Solar, Alippi2011A, Malaver2014Development, Rojas2015Design}.
%Considering the intermittent power supply, efforts have to be made to ensure the reliability and efficiency of the system under frequent power failures.
%, to guarantee the progress of tasks.

In a normal system, intermittent power supply brings unpredictable outages and will end the progress because of execution state loss.
Previous works provide static and dynamic checkpointing strategies allowing system to roll back to a preformed checkpoint and continue the uncompleted program.
Static checkpointing strategy presets checkpoints in the programs before real-time execution.
Mementos~\cite{ransford2012mementos} provides an automatic checkpoint placing strategy to maximize task progress.
DINO~\cite{} tackles the data consistency problem by placing checkpoints to construct transactional and atomic operations. 
Dynamic checkpointing strategy utilizes hardware modifications to enable run-time checkpointing to minize rollback distance. 
Hibernus~\cite{balsamo2015hibernus} and QuickRecall~\cite{} utilizes voltage detector to trigger software state backup in case of power failures. 
NVP~\cite{wang20123us} proposes control circuit to realize data migration between processor and FeRAM  memory according to supply voltage changes.
In conclusion, static checkpointing can solve the data consistency problem dispense with hardware modifications in TPC, but frequent rollbacks will significantly slow down the system forward progress~\cite{Ma2015Architecture}. 
On the other hand, dynamic checkpointing with hardware backup mechanism optimize the efficiency of system recovery, but cannot support atomic operation recovery with passive backup trigger~\cite{Alpaca}.  



NVIO~\cite{li2016hw}


Although the above mentioned checkpointing strategies realizes efficient and safe intermittent execution of computing tasks on a processor, none consider the recovery of peripheral(I/O) devices.
Peripheral brings two main challenges to a transiently powered system.
Firstly, a peripheral needs to be initialized and well configured before execution. 
NVIO~\cite{li2016hw} tries to solve the problem by replace volatile DFFs to nonvolatile FeRAM based NVFF. 
However, the extreme wide variety of peripherals types and producers make it unrealistic to implement NVFF replacements to all peripherals. 
%Furthermore, some peripherals cannot be completely recovered by simple restoring peripheral states~\cite{}. 
Secondly, peripheral actuations are mostly concurrent atomic operations~\cite{LuciaSurvey} causing consistency issues.
Moreover, the atomicity of peripherals require static checkpointing strategy which is conflict with the most efficient but inflexible NVP and leads to low efficiency.
Alpaca~\cite{lucia2017Alpaca} proposes a checkpointing-free peripheral supporting recover framework by enabling peripheral operation recovery in a single control thread, but lack of discussion on multiple concurrent threads recovery in intermittent power scenario.

% Introduce the newly presented design methodology.
Targeting the recovery problem of TPC with multiple devices, this paper proposes REMARK, a \underline{R}eliable and \underline{E}fficient \underline{M}ulti-device Recovery Fr\underline{A}mewo\underline{RK}, containing novel hardware components on top of NVP and corresponding software optimizations.
REMARK proposes a hybrid checkpointing strategy by combining safe static checkpointing and highly efficient hardware based dynamic checkpointing in NVP to achieve the target of both efficiency and reliability.
To accelerate the recovery procedure of peripherals, REMARK provides a peripheral configuration tracking strategy and hardware recovery modules.
The major contributions are listed as follows.

%
\begin{itemize}
    \item This paper addresses the problem of efficient concurrent execution of peripherals in intermittent power scenarios.

    \item REMARK hides the complexity associated with peripheral recovery by automatic peripheral configuration tracking strategfy and efficient recovering architecture. It maintains easy-to-configure interfaces for TPC designer such that they can focus on the functionality instead of peripheral implementation details. 

    \item We implement a REMARK-enabled NVP chip to combine the static checkpoint pre-placement and the efficient dynamic run-time checkpointing. A real application is evaluated on this platform and the results show that REMARK can speed up the data transmission tasks by $13\times$ compared with state-of-the-art solutions. 

    \item We further analyze the impact of different design parameters of REMARK on a sofware simulator, NVnodeSim. Based on these analyses, program design guidelines are proposed to improve the resistance against deadlock and reduce the timing overhead by as much as 36.5\%.
\end{itemize}

The rest of this paper is organized as follows.
Sec.~\ref{sec:motivation} illustrates the challenges to design a recovery strategy for TPCs with multiple peripherals.
Sec.~\ref{sec:system} presents system overview of REMARK, which contains the hardware architecture, the offline multi-device checkpointing strategy and the online recovery mechanism.
The details of these three components are presented in Sec.~\ref{sec:hardware},~\ref{sec:offline}, and~\ref{sec:online}, respectively. 
Sec.~\ref{sec:implementation} presents the fabrication and the performance of a REMARK-enabled NVP chip.
Sec.~\ref{sec:evaluation} explores the impact of different design parameters in REMARK with a software simulator and summarizes design rules for TPC systems. .
The remaining sections introduce the related works and conclude this paper.
