\section{Introduction} \label{sec:introduction}
%   1.5pages
% Transiently Powered Systems and Sustaining computing.
Transiently Powered Computers (TPCs)~\cite{Ma2015Architecture, ransford2013transiently, ransford2012mementos} are a new class of battery-less embedded systems that depend solely on energy harvested from ambient environment.
TPC systems are especially attractive to many IoT applications since it is often challenging to employ traditional batteries where it is inconvenient, costly or even dangerous to replace or service batteries. 
Energy harvesting can eliminate the need for batteries or wires and enable long-term operation of these systems with little maintenance.
However, design of these systems faces unique challenges since the power supply is intrinsically unstable and affected by the ambient environment conditions~\cite{Ma2015Architecture, ransford2013transiently,wang2014mppt}.
For example, the reliability of solar powered systems largely depends on the weather, location and environmental changes of different applications~\cite{Abas2015Solar, Alippi2011A, Malaver2014Development, Rojas2015Design}.
%Considering the intermittent power supply, efforts have to be made to ensure the reliability and efficiency of the system under frequent power failures.
%, to guarantee the progress of tasks.

Considering the intermittent power supply, how to bridge TPC executions across power failures becomes essential. 
It is known that a volatile processor will lose its state after power failure, and frequent rollbacks will significantly slow down the system forward progress~\cite{Ma2015Architecture}. 
Several checkpoint and rollback strategies have been proposed to save the processor state and improve the forward progress.
Mementos~\cite{ransford2012mementos}, Hibernus~\cite{balsamo2015hibernus} and QuickRecall~\cite{jayakumar2014quickrecall} provide automatic software checkpointing mechanisms for volatile processors (VPs).
Meanwhile, non-volatile processor (NVP)~\cite{wang20123us} and NVIO~\cite{li2016hw} are proposed to automatically store the processor and I/O state more efficiently based on hardware.

However, these schemes mainly aim at recovering computation tasks on a processor. 
None of them considers the unique challenges associated with the recovery of multiple peripheral devices.
Firstly, peripherals contain analog circuits and interactive operations with environment, making it impossible to backup the device state during peripheral operations.
Second, there are a wide variety of volatile peripherals produced by different manufacturers and it is difficult to apply nonvolatile memory to all peripherals. 
Third, a complete TPC system runs computation tasks on the processor and peripheral operations on peripheral devices simultaneously, where consistency issue arises.
Therefore, recovering these peripherals faces more reliability and efficiency challenges. 
Previous work~\cite{Xie2015,van2016intermittent,Lucia2015} have noticed the data consistency problem between processor and off-chip NVM,  a special case of peripherals. 
Since peripherals are generally volatile and their states cannot be checkpointed anywhere like a processor, a systemically recovery framework is critical for a multi-device TPC system.

% Introduce the newly presented design methodology.
Targeting the recovery problem of TPC with multiple devices, this paper proposes REMARK, a \underline{R}eliable and \underline{E}fficient \underline{M}ulti-device Recovery Fr\underline{A}mewo\underline{RK}, containing novel hardware components on top of NVP and corresponding software optimizations.
REMARK enhances the functionality of NVP with four new hardware modules to support efficient multi-device checkpointing.
The software optimizations include an offline program transformer and an online recovery mechanism, which enable efficient and reliable resumption of multiple peripherals.
The major contributions are listed as follows.

%
\begin{itemize}
    \item To the best of authors' knowledge, REMARK is the first work that systematically addresses reliable and efficient peripheral operations in multi-device TPC systems.

    \item REMARK hides the complexity associated with multi-device recovery by efficient hardware architecture and software/runtime supports. It maintains easy-to-configure interfaces for TPC designer such that they can focus on the functionality instead of peripheral implementation details. 

    \item We implement a REMARK-enabled NVP chip and the supporting hardware platform. A real application is evaluated on this platform and the results show that REMARK can speed up the data transmission tasks by $13\times$ compared with state-of-the-art solutions. 

    \item We further analyze the impact of different design parameters of REMARK on a sofware simulator, NVnodeSim. Based on these analyses, program design guidelines are proposed to improve the resistance against deadlock and reduce the timing overhead by as much as 36.5\%.
\end{itemize}

The rest of this paper is organized as follows.
Sec.~\ref{sec:motivation} illustrates the challenges to design a recovery strategy for TPCs with multiple peripherals.
Sec.~\ref{sec:system} presents system overview of REMARK, which contains the hardware architecture, the offline multi-device checkpointing strategy and the online recovery mechanism.
The details of these three components are presented in Sec.~\ref{sec:hardware},~\ref{sec:offline}, and~\ref{sec:online}, respectively. 
Sec.~\ref{sec:implementation} presents the fabrication and the performance of a REMARK-enabled NVP chip.
Sec.~\ref{sec:evaluation} explores the impact of different design parameters in REMARK with a software simulator and summarizes design rules for TPC systems. .
The remaining sections introduce the related works and conclude this paper.
