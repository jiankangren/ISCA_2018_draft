\section{Introduction} \label{sec:introduction}
%
By 2020, 50 billion Internet of Things (IoTs) devices are expected to be connected to the Internet~\cite{nordrum2016popular,cisco2013IoT}.
By 2025, these connected IoTs are predicted to generate more than 11000 billion dollars economic value~\cite{manyika2015unlocking}.
In some domains, the IoTs have already exhibited tremendous potential to bring new opportunities with it, for instance, 
the wearable devices~\cite{metcalf2016wearables}, the smart home~\cite{ho2016smart}, the smart industry~\cite{shariatzadeh2016integration}, and the smart meters~\cite{azariadi2016ecg,haghi2017wearable}.

With more application scenarios deployed and more devices connected to the Internet, a vital challenge is observed to limit
the wider application of IoTs - the battery life.
To begin with, it is inconvenient to charge the batteries in some scenarios. 
Today, most of the wearable devices needs to be charged once a day or several days~\cite{}.
In addition, renewing batteries in huge amount like those IoTs in smart industry~\cite{wang2016towards} or in very distributed area like environmental and infrastructure monitoring~\cite{} can
very costly, and sometimes unrealistic.
What's more, the existence of batteries significantly frustrates application with volume and / or safety concerns, for example in the implantable devices~\cite{}.

Energy harvesting techniques emergie as one of the promising techniques to solve problems brought by batteries. By harvesting energy from ambient environment, including
solar, vibration, RF and thermal energy, the lifetime of IoTs can be eventually infinite. This can essentially extend the application domain of IoTs, for instance,
smart parking~\cite{}, environment monitor~\cite{}, smart argriculture~\cite{}, implantable devices~\cite{} etc.

While a coin has two sides, the energy harvesting can provide only unstable power supplies, depending on the application scenarios~\cite{}.
In the processor level, various techniques have been proposed to mitigate the inconsistency problem brought by the unstable power supply.
These techniques includes software checkpointing under guidance of programming languages or compliers~\cite{}, and hardware based architectural supports~\cite{}.
Although the processor for signal processing is in same extend mitigated, the energy and time distribution of processing data actually takes only very small part.
For instance, the signal processing on processor takes less than 20\% in every wake-up cycles, while peripherals initializations contribute to over 80\%~\cite{}.
Similar observations can be found in really deployed systems: 
WISPcam (image sensor + RF + memory)~\cite{}, Implant cochlear (acoustic encoder + demodulator + sound sensor)~\cite{}, Mini-satellite (RF + thermometer + gyroscope + magnetometer + gas sensors)~\cite{}.
Optimizating consistency with processor alone can not solve the system level challenges in energy harvesting systems.
The peripheral recovery is another critical problem in energy harvesting scenarios and remaining unsolved, and is an important but last step to bring real energy 
harvesting systems to life[Cite Brandon-SNAPL17,Maeng-OOPSLA17 ].

Modifying the architecture of all peripherals in a system to support unstable power supply can mitigate the inconsistency problem within the peripherals, but the data transmission between the processor
and the peripherals remains unstable. 
More importantly, peripheral actuations are mostly concurrent atomic operations~\cite{Lucia2017}, which requires static checkpointing, conflicting with the dynamic NVPs based solutions. Directly grafting NVP techniques to peripherals to be nonvolatile is rough and inefficient.
In addition, considering the amount of various peripherals in the market, it is not time and cost efficient to make all the peripherals nonvolatile.


% Introduce the newly presented design methodology.
Targeting at the recovery of a system with multiple peripherals, we propose REMARK, a \underline{R}eliable and \underline{E}fficient \underline{M}ulti-device Recovery Fr\underline{A}mewo\underline{RK}, for hardware and software co-optimization on top of a NVP to address the system level issues.


The major contributions are listed as follows.

%
\begin{itemize}
    \item This paper addresses the problem of concurrent execution of peripherals in intermittent power scenarios, by proposing a hybrid checkpointing strategy, combining static checkpointing for peripherals and efficient dynamic checkpointing in NVP to achieve both reliability and efficiency.

    \item To accelerate the recovery procedure of peripherals, for the first time REMARK provides a peripheral configuration tracking strategy and according methdology for hardware-assisted recovery modules, which simplifies the peripheral recovery through an easy-to-configure interfaces. 

	\item We analyze the design tradeoffs in REMARK to improve the resistance against deadlock and reduce the timing overhead by as much as 36.5\%.

    \item To verify the architecture, a REMARK-enabled chip is fabricated. Furthermore a real application is evaluated and the results show that REMARK can speed up the data transmission tasks by $13\times$ compared with traditional solutions. 

\end{itemize}

%The rest of this paper is organized as follows.
%Sec.~\ref{sec:motivation} illustrates the challenges to design a recovery strategy for TPCs with multiple peripherals.
%Sec.~\ref{sec:system} introduces the hybrid checkpointing strategy and the software/hardware framework overview of REMARK.
%The details of these three components are presented in Sec.~\ref{sec:hardware} and Sec.~\ref{sec:offline}, respectively. 
%Sec.~\ref{sec:implementation} presents the fabrication and the performance of a REMARK-enabled NVP chip.
%Sec.~\ref{sec:evaluation} explores the impact of different design parameters in REMARK with a software simulator and summarizes design rules for TPC systems.
%The remaining sections conclude the related works and this paper.

%\bibliographystyle{ieeetr}
%\bibliography{references}
